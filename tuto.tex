%!TEX encoding = UTF-8 Unicode

\documentclass[a4paper]{article}

\usepackage[french]{babel}
\usepackage[utf8]{inputenc}
\usepackage[T1]{fontenc}
\usepackage{tikz}
\usepackage{pgfplots}
\usepackage{amsmath} %package mathematique
\usepackage{graphicx}
\usepackage{lipsum}
\usepackage{multido} % scripting code latex
\usepackage{listingsutf8}
\usepackage{xcolor}

\pgfplotsset{compat=1.13}

% Nouvelle commande qui utilise le multido
\newcommand{\emptyLines}[1]{%
\multido{\iN=1+1}{#1}{\phantom{}\dotfill\newline}
}

% Definition des parametres pour les listings
\lstset{%
backgroundcolor=\color{blue!05!white},
basicstyle=\normalsize,
commentstyle=\color{green!40!black},
keywordstyle=\color{blue!90!black},
morekeywords={},
frame=single,
numbers=left,
inputencoding=utf8,
extendedchars=true,
literate={á}{{\'a}}1 {à}{{\`a}}1 {ã}{{\~a}}1 {é}{{\'e}}1 {è}{{\`e}}1 {ê}{{\^e}}1 {ô}{{\^o}}1
}

\begin{document}

\tableofcontents

\section{Lipsum - Génération de texte}

\lipsum[2] % 2 paragraphes

\section{\'Equations}

\subsection{Racines carrées et fractions}

Voici un ensemble d'exemple d'équations.

\begin{equation}
\sqrt{3\times x+\left(5+\frac{5}{4}\right)}
\end{equation}

\begin{equation}
\frac{5}{\left(1+3\right)} \Rightarrow 5\times\frac{1}{4}
\end{equation}

\begin{equation}
	\dfrac {\sqrt {\dfrac {\left( 2x+5\right) ^{2}}{y+2}}}{5}=x
\end{equation}

\subsection{Puissances et indices}

\begin{equation}
2^{2+4} = 4
\end{equation}

\begin{equation}
2_{5+x}
\end{equation}

\begin{equation}
2^{a}_{b}
\end{equation}

\begin{equation}
\stackrel{a}{\longleftrightarrow}
\end{equation}

% Avec le package mathématique

\begin{equation}
\overset{a}{\longleftrightarrow}
\end{equation}

\begin{equation}
\underset{a}{\longleftrightarrow}
\end{equation}

% La c'est encore mieux :p

\begin{equation}
\frac{1}{n} \xrightarrow[n \rightarrow \infty]{} 0
\end{equation}

\subsection{Références}

\begin{equation}\label{Equation de base}
e^{i\pi}+1=0
\end{equation}

Voici une référence à l'équation précédente (\ref{Equation de base}).

L'équation est accessible à la page \pageref{Equation de base}.

\section{Affichage de functions}

\begin{tikzpicture}
\begin{axis}[
	ymax=10,
	grid=major, 
	legend cell align = center,
	title=Deux fonctions affichées]
	\addlegendentry{x};
	\addlegendentry{y};
	\addplot[blue,samples=200]{x^2};
	\addplot[red,domain=0:20,samples=500]{sqrt(x)};
\end{axis}
\end{tikzpicture}

\section{Images}

Voici la première image :
\begin{figure}[h] % indique à latex de positionner la figure ou il souhaite ou pas (h = here) t = top, b = botton, p = group figures
\begin{center}
\includegraphics[height=2cm]{debian-logo}
\end{center}
\end{figure}

% Autre mehode
%\centering\includegraphics[height=2cm]{debian-logo} \\

\section{Tableaux}

\begin{table}[!h] % Je veux la table ici
\begin{center}
	
\begin{tabular}{|c|c|c|}
	\hline
	\multicolumn{3}{|c|}{Titre} \\ \hline
	1	&	2	&	3 \\ \hline
	4	&	5	&	6 \\ \hline
\end{tabular}

\caption{tableau d'exemple}
\end{center}
\end{table}

\section{Minipages}
\fbox{
\begin{minipage}[c]{\dimexpr 0.5\textwidth-2\fboxsep-2\fboxrule-0.25cm \relax}
	\lipsum[2]
\end{minipage}}% Pour supprimer l'espace
\hspace{0.5cm}%
\fbox{
\begin{minipage}[c]{\dimexpr 0.5\textwidth-2\fboxsep-2\fboxrule-0.25cm \relax}
	\lipsum[2]
\end{minipage}}

\section{Multido}

Texte à remplir :

\emptyLines{10}

\section{Code}

\begin{lstlisting}[language=C,title=Code C,captionpos=b]
#include <stdio.h>
	
/* Method main */
	
int main(int argc, char** argv) {
	return 0;
}
\end{lstlisting}

\lstinputlisting[language=C, title=exemple.c, captionpos=b]{exemple.c.txt}


\section*{Annexes} % Non présent dans la toc



\end{document}
